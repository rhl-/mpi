\documentclass{isocpp_proposal}
\usepackage{hyperref}

\documentnumber{$\pi$}
\date{\today}
\project{C++ Message Passing Standard Proposal}
\replyto{Ryan H. Lewis $\langle$ me@ryanlewis.net $\rangle$}

\begin{document}
\maketitle

\section{Introduction}
The purpose of this document is to outline new language features to translate 
C++ data structures from the address space of one process to the address spaces of other processes, possibly on different physical machines. 
Presently there is no message passing support in C++. 

\subsection{Need \& Prior Art}
There is a definite demand for distributed memory programming. Distributed memory programming is of the upmost necessity for 
large scale simulations and data analysis. The Message Passing Interface is an existing C/Fortran standard with a similar goal \cite{mpistandard}.
The standard defines an MPI Type system, similar to the type system inherent in C++. It does this to provide runtime type introspection. 

Separately, companies such as Apache and Google have found MapReduce to be useful for parallelism. The map reduce framework is  distributed memory model. 
\href{http://www.open-std.org/jtc1/sc22/wg21/docs/papers/2012/n3446.pdf}{Proposal N3446} is an existing proposal to add MapReduce into C++. In this proposal
they mention the possibility of a distributed memory map() and reduce().

Existing distributed memory libraries such as map reduce frameworks implement a subset of a broader set of message passing algorithms, while broader standards
such as MPI do not integrate with the C++ typing system. This shifts added complexity onto the library user to communicate \emph{how} to communicate data structures.

\section{Motivation and Scope}

\subsection{Why is this important? What kinds of problems does it address?}
The motivation for the addition of message passing to the C++ standard is to enable 
portable, efficient, general purpose distributed memory algorithm development. 

\subsection{ What is the intended user community?}
The intended community is anyone interested in using more memory than is available on one machine.

\section{Impact On the Standard}
A message passing component in the standard would rely first and foremost on a networking standard for implementation, and additionally a reflections standard.

\section{Design Decisions}

Why did you choose the specific design that you did? What alternatives did you consider, and what are the tradeoffs? What are the consequences of your choice, for users and implementers? What decisions are left up to implementers? If there are any similar libraries in use, how do their design decisions compare to yours?

\section{Technical Specifications}

The committee needs technical specifications to be able to fully evaluate your proposal. Eventually these technical specifications will have to be in the form of full text for the standard or technical report, often known as "Standardese", but for an initial proposal there are several possibilities:

Provide some limited technical documentation. This might be OK for a very simple proposal such as a single function, but for anything beyond that the committee will likely ask for more detail. 
 
Provide technical documentation that is complete enough to fully evaluate your proposal. This documentation can be in the proposal itself or you can provide a link to documentation available on the web. If the committee likes your proposal, they will ask for a revised proposal with formal standardese wording. The committee recognizes that writing the formal ISO specification for a library component can be daunting and will make additional information and help available to get you started.
 
Provide full "Standardese." A standard is a contract between implementers and users, to make it possible for users to write portable code with specified semantics. It says what implementers are permitted to do, what they are required to do, and what users can and can't count on. The "standardese" should match the general style of exposition of the standard, and the specific rules set out in 17.5, Method of description (Informative) [description], but it does not have to match the exact margins or fonts or section numbering; those things will all be changed anyway.
\section{Acknowledgements}

\section{References}
\begin{thebibliography}{1} %1 is the number of entries below.. one can also replace with bibtex..
\bibitem{mpistandard} Message Passing Interface Forum, {\em \href{www.mpi-forum.org/docs/mpi-3.0/mpi30-report.pdf}{The Message Passing Standard Version 3.0}},  2012.
\end{thebibliography}

\end{document}  
