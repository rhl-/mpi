\documentclass{isocpp_proposal}

\documentnumber{$\pi$}
\date{\today}
\project{C++ MPI Standard Proposal}
\replyto{Ryan H. Lewis $\langle$ me@ryanlewis.net $\rangle$}

\begin{document}
\maketitle

\section{Introduction}

A very brief high level view of your proposal, understandable by C++ committee members who are not necessarily experts in whatever domain you are addressing.

\section{Motivation and Scope}

Why is this important? What kinds of problems does it address? What is the intended user community? What level of programmers (novice, experienced, expert) is it intended to support? What existing practice is it based on? How widespread is its use? How long has it been in use? Is there a reference implementation and test suite available for inspection?

\section{Impact On the Standard}

What other library components does does it depend on, and what depends on it? Is it a pure extension, or does it require changes to standard components? Can it be implemented using C++11 compilers and libraries, or does it require language or library features that are not part of C++11?

\section{Design Decisions}

Why did you choose the specific design that you did? What alternatives did you consider, and what are the tradeoffs? What are the consequences of your choice, for users and implementers? What decisions are left up to implementers? If there are any similar libraries in use, how do their design decisions compare to yours?

\section{Technical Specifications}

The committee needs technical specifications to be able to fully evaluate your proposal. Eventually these technical specifications will have to be in the form of full text for the standard or technical report, often known as "Standardese", but for an initial proposal there are several possibilities:

Provide some limited technical documentation. This might be OK for a very simple proposal such as a single function, but for anything beyond that the committee will likely ask for more detail. 
 
Provide technical documentation that is complete enough to fully evaluate your proposal. This documentation can be in the proposal itself or you can provide a link to documentation available on the web. If the committee likes your proposal, they will ask for a revised proposal with formal standardese wording. The committee recognizes that writing the formal ISO specification for a library component can be daunting and will make additional information and help available to get you started.
 
Provide full "Standardese." A standard is a contract between implementers and users, to make it possible for users to write portable code with specified semantics. It says what implementers are permitted to do, what they are required to do, and what users can and can't count on. The "standardese" should match the general style of exposition of the standard, and the specific rules set out in 17.5, Method of description (Informative) [description], but it does not have to match the exact margins or fonts or section numbering; those things will all be changed anyway.
\section{Acknowledgements}

\section{References}
\begin{thebibliography}{1} %1 is the number of entries below.. one can also replace with bibtex..
\bibitem{key} John Doe {\em Title and stuff}  2013.
\end{thebibliography}

\end{document}  
